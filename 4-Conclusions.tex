\chapter{Выводы}
\label{conclusions}

В последние годы вопрос о том, по какой траектории происходит развитие нефтегазового комплекса, как и всей энергетической системы, приобретает все больший интерес, как со стороны экспертов, так и со стороны широкой общественности \cite{bakhtin2017trend,kuzminov2017global}. 
Этому способствует несколько факторов. 


\begin{itemize}
\tightlist
\item Во-первых, темпы экономического развития приводят к значительному росту мирового энергопотребления. Как отмечается в докладе Аналитического Центра при Правительстве РФ , значительный рост потребления энергоресурсов происходит за счёт развивающихся стран, преимущественно Азиатско-тихоокеанского региона, в то время как в развитых странах объем выработки электроэнергии стабилен, а динамика потребления схожа с тенденциями общеэкономических приростов и спадов. 

\item Во-вторых, наблюдается изменение структуры запасов углеводородов. Как отмечается в «Энергетической стратегии России на период до 2035 года»  (сформулированной в 2015 году), отечественная нефтяная отрасль сталкивается с такой проблемой, как «увеличение себестоимости добычи вследствие преобладания труднодоступных запасов нефти (далее по тексту ТРиЗ) и большой выработанности действующих месторождений, что усложняет удержание достигнутых уровней добычи нефти». При этом одной из задач, ставящейся перед нефтяным сектором, является освоение ТРиЗ в объёмах до 17\% от общей добычи, которая может быть решена путём развития добывающих технологий. 

\item Наконец, в-третьих, всё большую роль в энергетическом секторе играют источники возобновляемой энергии (т.н. ВИЭ), что сказывается на структуре энергетических рынков . Эксперты, политики и граждане всё больше озабочены экологическими и климатическими вызовами, что свидетельствует о необходимости диверсификации энергоносителей. Дополнительно стоит отметить негативное влияние внешних экономических и политических ограничений на сырьевой сектор российской экономики.
\end{itemize}

Таким образом, энергетическая сфера находится в процессе постоянной трансформации, а одним из важных вопросов повестки дня нефтяного сообщества является оптимизация методов геологоразведки, добычи и использования энергоносителей. 

Анализировать, по какой траектории движется изменение научно-технических и технологических процессов нефтедобычи, можно несколькими способами. Наиболее очевидным видится опрос экспертов, специализирующихся на вопросах добычи. 

Методы экспертных опросов (также называемые методами экспертных оценок) широко используются в различных исследованиях, в которых невозможны или труднодоступны другие формы исследований ввиду отсутствия объективных данных. Таким образом реализуется подавляющее большинство форсайт-исследований. К достоинствам экспертного опроса можно отнести их относительную простоту и доступность, а также возможность применения в случае отсутствия информации об изучаемом явлении. 

В то же время очевидным недостатком экспертного опроса являются возможный субъективизм и ограниченность экспертов, их приверженность определенной точке зрения. Как отмечается в работе Бахтина с соавторами \cite{bakhtin2017trend}, в течение последних лет объёмы экспертно-аналитической и научной литературы, а также информации в целом, стремительно растут (по некоторым оценкам объёмы информации удваиваются каждые два года), так что задача получения, фильтрации, переработки и рефлексивного восприятия всей информации становится фактически невозможной. При этом эксперту необходимо развиваться и совершенствоваться в различных содержательных направлениях, что требует ещё больших трудовых и временных инвестиций. Это свидетельствует о необходимости разработки и формирования дополнительной обратной связи, которая призвана помочь экспертному и профессиональному сообществу анализировать огромные объемы информации и выделять из нее наиболее релевантные аспекты, в частности – выявлять технологические тренды. 

С развитием автоматизированных методов обработки неструктурированных данных, в частности текстовых данных, популярность набирает тематическое моделирование научных текстов \cite{blei2006dynamic}. Как было продемонстрировано в работе Блея и Лафферти, тематическое моделирование оказывается перспективным инструментов отслеживания трендов в таких научных направлениях, как ядерная физика и нейронауки \cite{Blei:2003}, технологии агропромышленного комплекса \cite{bakhtin2017trend} и так далее. Изучение автоматически выделенных тематик во временной перспективе иллюстрирует изменение интереса научного сообщества к различным объектам и предметам исследования. Достоинством этого метода является возможность автоматизированной обработки огромных массивов информации и выявления латентных (скрытых) тематик текстов. 
При этом тематическое моделирование нельзя назвать исключительно автоматизированным методом, так как полученные в результате машинной классификации тематики в дальнейшем должны быть вручную просмотрены и проработаны экспертами-специалистами предметной области. Таким образом, тематическое моделирование может рассматриваться как метод, заключающий в себе достоинства и автоматизированной обработки текста, и экспертной оценки. Реализация подобного метода в приложении к различным содержательным задачам позволит сформировать диалог между наукой и стратегией на принципиально новом уровне.
% 
Тематическое моделирование позволяет оперативно обрабатывать значительные объемы текстов для сужения найденных понятий до небольших значимых фрагментов текста - топиков. Каждый топик
представляется набором слов и от качества этого представления зависит возможная интерпретация.

Автор показал результативность подхода к улучшению интерпретируемости тематик на основе последовательной регуляризации. 

Примененные методы управление отношением «плотность-разрежённость» открывают возможности настройки модели на предметную область текстов. Автор показал принципы создания и настройки модели тематик, которые позволяют вести интеллектуальный поиск (разведку) высоко сфокусированныхисточников знаний.

Кластеризация топиков была проверена с помощью двух методов для векторизации слов (FastText, GloVe) и двух методов для уменьшения размерности векторного пространства (TSNE, MDS). Результаты представлены в виде диаграмм и уверено показывают наличие кластеров.

Подход к анализу текстовой информации на основе моделирования тематик широко используется во внутренних процессах компании ООО «Газпромнефть НТЦ» для оптимизации процессов управления знаниями, выявления наиболее перспективных направлений исследований и поиска opinion leaders в определенных научных направлениях.

Важно отметить, что выбранный автором метод показал высокую скорость анализа, что делает его возможным для применения в онлайновых процессах поиска. Например, на сайте издательства в качестве средства улучшающего поиск и дающего рекомендации читателям по статьям со схожей тематикой. 

Также следует отметить, что разработанная методика может быть в дальнейшем усовершенствована и адаптирована для анализа существенно больших массивов динамических данных и выделения ключевых направлений технологического развития как в более широких, так и в более узких областях. 

Cуществующие прогнозы научно-технического развития (с том числе форсайт-прогнозы) в большинстве своем экстраполируют существующие тренды на долгосрочную перспективу. Таким образом, большой интерес приобретают работы, в которых становится возможным выявление новых технологических направлений, способных существенно видоизменить структуру рынков.

Сами по себе отдельные технологии не следует рассматривать как оторванные и изолированные друг от друга инициативы. В действительности многие технологические направления развиваются параллельно, что является результатом венчурной политики, технологического развития и других сопутствующих факторов. 

Ввиду этого важным направлениям анализа технологических трендов выглядит изучение коэволюции развития сразу нескольких технологий. Именно изучение совокупности научно-технических инициатив позволит содержательно проанализировать направление технологического развития.

%%%

В раделе \ref{sec:di} автором представлен новый взгляд на процесс публикации научных статей. Определены показатели продуктивности и стратегии управления продуктивностью процесса публикаций.
 
Организационная среда должна служить инструментом для повышения эффективности основных производственных процессов. Признание научно-исследовательской организацией того факта, что публикация научных статей является одним из основных производственных процессов означает, что необходимо создавать специальные подразделения, нацеленные на поддержку эффективности этого процесса. Мерой зрелости процесса служит степень разделения труда его участников. Учёный должен заниматься своими прямыми обязанностями - исследованиями и не обязан вникать в детали процессов оформления командировок, эргономичности презентаций и тонкостей общения с издателями и т.п.

Автором разработана ролевая модель, которая позволит разгрузить учёных от формальных трудозатрат по публикации результатов исследований и в некоторых случаях избежать появления «гостевых» соавторов.

Из-за ограничения по объёму публикаций в выпуске издателя организациям необходимо расширять список издательств, в которых публикуются их исследователи, чтобы поддерживать темп роста количества опубликованных статей. 

Показатель продуктивности выражающий долю отвергнутых издательством статей является важной характеристикой процесса публикации результатов исследований не только на организационном, но и на отраслевом уровне. Возможность анализа этого показателя позволяет оценить достаточность ёмкости рынка научных издательств и степень конкуренции за публикацию в изданиях с высоким импакт-фактором. 

%%

В разделе \ref{sec:scrum} автором обобщена и проработана формализация процесса самоорганизации команд для достижения определённой цели -- написания научных статей. 
В исследовании разработан детальный алгоритм образования графа соавторств широко используемого в различных исследованиях.
Сформулированы основные теоретические утверждения, даны определения \emph{укомплектованности команды} и \emph{несостоявшейся научной статьи}. 
Сформулирована гипотеза (Гипотеза \ref{hyp:ex1}) об инвариантность графа соавторства относительно введения Scrum ролей в процесс написания статей.
В результате проведённого автором оптимизационного эксперимента найдены оптимальные значения параметров для построенной  модели написания статей. 
По результатам, сделанным на оптимизированной модели соавторства разработанной автором, эффект от введения гибких методик (Scrum) в процесс написания научных статей небольшими командами соавторов состоит в следующем:

\begin{itemize}
\tightlist
\item  Среднее время написания научной статьи ($T_{pub}$) не  изменяется
\item  Средняя доля несостоявшихся научных статьей   ($Frac_{notpub}$) уменьшается
\end{itemize}

Общее влияние Scrum на процесс написания научных статей командой соавторов является положительным. 
То, что $T_{pub}$ не изменяется может служить экспериментальным подтверждением Гипотезы \ref{hyp:ex1}.

Продуктивность команд, образованных по комплементарному принципу, становится выше от применения гибких методик и Scrum, в частности.

%%%

В эксперименте описанном в разделе \ref{seq:emo} подтверждена гипотеза о возможности выделения эмоционально-окрашенных фрагментов текста из научных статей. Научные статьи используют академическую лексику и ожидать в них градус эмоций сравнимый с отзывами на кинофильмы было бы наивно. Но современные концепции обработки текста, основанные на анализе контексте, позволяют выделять и классифицировать изменения эмоциональности достаточно точно для того, чтобы обрабатывать даже научные статьи. Автор считает, что проведённое исследование открывает возможности по созданию дополнительных инструментов для аннотации и классификации научных текстов. 

Наилучшее оценку по качеству выделения эмоционально окрашенных фрагментов текста показали рекуррентные нейронные сети. Точность по метрике Accuracy для них составила 88\%. Важно отметить, что по скорости обучения рекуррентные сети существенно проигрывают свёрточным сетям. Автор видит объяснение разности в производительности в том, что для обучения сверточных нейронных сетей возможна более высокая степень параллельных вычислений. Тогда как для рекуррентных нейронных сетей необходимо поддерживать последовательность предыдущих состояний нейронов. 

В дальнейших исследованиях автор планирует исследовать применимость эмоционально окрашенных фрагментов текста для задач классификации текстов в качестве признаков. Так же на взгляд авторов, научный интерес представляет анализ синтаксиса эмоционально окрашенных фрагментов текста.

%%%%

В эксперименте описанном в разделе \ref{sec:allo} проведён анализ динамики графа соавторства для одной организации на основании публичных данных о публикациях. Основным аналитическим инструментом был выбран двудольный граф соавторства, методически обоснованный автором в разделе \ref{sec:coath}. 

В работе применён много компонентный подход к прогнозированию изменению свойств графа соавторства. Анализ малых связанных компонент позволил выявить их долю в ежегодном увеличении количества авторов. Отметим, что доля малых компонент в рассматриваемом графе соавторства уменьшается со временем, что является структурным ограничением роста рассматриваемой организации.  

В 2016 году обнаружен «Эффект локтя» - резкое усложнение характера роста графа соавторства по годам.  
Автором сделан прямой прогноз роста на основании тренда роста авторов по годам и уточненный прогноз роста графа соавторства на основе моделирования с помощью методов машинного обучения. 

Проведённое сравнение точности классификаторов определило классификатор на основе нейронной сети как наиболее точный для данной задачи. 

Прогноз, сделанный на основе модели, показал результат (467) существенно меньший чем результат на основе тренда (585). 

В результате проведённого исследования автор сделал вывод о наличии в структуре графа соавторств важной информации о развитии графа соавторств, которая определяет прогноз роста. Что позволяет определить значимые признаки образования новых коллабораций, а также регрессионного предсказания новых связей между уже сформировавшимися исследовательскими коллективами. 

Использование методов векторизации графовых моделей в комбинации с извлечением признаков позволит улучшить точность предсказания появления новых связей, а также качественно измерить публикационную активность на основе публично доступных метрик журналов и конференций .

Так же автором предложен метод выделения направлений научных исследований на основе графа соавторства. Содержательно предложенный метод относится к top-down алгоритмам кластеризации. В качестве критерия выделения кластеров выбрана метрика \textit{Betweenness centrality}. 

В качестве критерия проверки качества кластеров выбрана метрика близости членов кластера и метрика удалённости различных кластеров на основе тематик научных статей, входящих в граф соавторства. 

Результатом применения предложенного метода является укрупненное виденье научных направления развития организации, сделанное на основе публичных данных о публикационной активности сотрудников.

Разработанный автором метод выделения направлений научных исследований на основе графа соавторства опробован на Научно-техническом центра ГазпромНефть. В результате выделены 16 кластеров, характеризующих деятельность организации.
Важными особенностями разработанного метода выделения направлений научных исследований на основе графа соавторства являются следующие:

\begin{itemize}
\tightlist
\item Рекурсивность алгоритма позволяет работать с графами различных порядков.
\item «Жадный» алгоритм определения качества кластеров позволяет корректировать оптимизацию на каждом шаге.
\item Применение двудольного построения графа соавторства позволяет анализировать различные проекции.
\item Работа на основании публичных данных даёт широкие возможности для применения в бизнес разведке.
\end{itemize}

Новизна предложенного автором метода выделения направлений научных исследований на основе графа соавторства состоит в использовании двудольного построения графа соавторства и в динамической модели кластеризации, использующей структурные метрики графа соавторства и метрики близости текстов научных статей.

%%%

Автор создал действующие модели движения персонала в организации и модель выполнения заданий. На основе взаимодействия этих моделей автор построил модель продуктивности, которая, отражает для научно-исследовательской организации изменения ИК. Экспериментальные результаты представлены в разделе \ref{sec:ic}.
 
Согласно мнению многих исследователей ИК сложно измерить. Предложенный автором драйвер ИК в виде производительности научно-исследовательской организации имеет самостоятельную ценность и характеризует ИК, как комплексный показатель организации.  

Автор построил зависимости ИК от различных времён адаптации новичков и различной сложности поступающих заданий, показали асимптотическое поведение ИК. Что позволяет моделировать ситуации разных видов задач, особенностей организации (текучесть, скорость адаптации, сложность задач и пр). 

В исследовании проанализировано как на ИК влияет нагрузка на персонал. Показано как со временем уменьшается продуктивность при высоких нагрузках и необходимости работать дольше 40 часов в неделю. Автором смоделирован эффекты «выгорания» и «усталости» персонала от длительной высокой нагрузки.

Результаты, полученные в исследовании, обладают научной новизной и практической ценностью, дают возможность детального исследования и моделирования динамики продуктивности.

%%%
Созданная автором и описанная в разделе \ref{sec:so} частная модель $\mathbb{M}_{GPN}$ организации оправдала себя как метод исследования социальных явлений и процессов организации посредством их воспроизведения в менее сложных формах и проведения необходимых операций с полученными таким образом аналогами реальных процессов в организационной среде.

Формальная математическая модель $\mathbb{M}_{\Omega}$ организации дает ответы на вопросы о ключевых компонентах деятельности по написанию и публикации научных статей.

Выбранный автором метод создания частной модели $\mathbb{M}_{GPN}$ показал результаты согласующиеся с эмпирическим исследованием публикаций конкретной научно-исследовательской организации.

Был проведен эксперимент по многоагентному симулированию, в котором в качестве агентов выступали научные сотрудники лабораторий, взаимодействующие друг с другом и производящие в качестве результата своей работы научные статьи. Создана частная модель $\mathbb{M}_{GPN}$ путем калибровки на данных НТЦ «Газпромнефть». В работе использовано программное обеспечение Anylogic.

На основании созданной частной модели  автор пришел к необходимости дальнейшего изучения чувствительности от следующих свободных параметров:
\begin{enumerate}
\tightlist
\item Максимальное количество соавторов
\item Среднее количество соавторов в статьях
\item Распределение количества соавторов
\item Количество статей в год на одного сотрудника
\end{enumerate}

Полученные в результате симуляционного эксперимента результаты согласуются с эмпирическими наблюдениями. 
Исходя из этого, автором сделан  вывод о том, что работа исследователей может быть смоделирована с использованием агентного подхода. 
Решение подобной задачи является важным шагом на пути к идентификации механизмов коллективной работы и формирования коллективного наукоемкого продукта.
